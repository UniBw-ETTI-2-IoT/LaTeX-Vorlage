\chapter{Einleitung}
\label{lab:einleitung}
Hier schreiben wir etwas zur Motivation und zum Aufbau der Arbeit und können schon mal eine Quelle zitieren. 
In \cite{Beispiel:1999} steht was Tolles geschrieben. 
\section{Motivation}
\section{Literaturverweise}\label{sec:literaturverweise}
%
Literaturverweise werden durch die Nutzung des cite-Befehls  \cite{ein_artikel} erzeugt.

Die Einträge in der .bbl Datei werden ausgehend von der Datei literatur.bib automatisch erstellt.
Diese kann händisch erstellt werden, es empfielt sich aber ein entsprechendes Tool zu verwenden (z.B. Jabref\footnote{\url{https://www.jabref.org/}} oder Citavi\footnote{\url{https://www.citavi.com/de}}).\\
Nur die mit verlinkten Elemente werden im Literaturverzeichnis angezeigt, alle übrigen nicht (um eine nicht verlinkte Ausgabe zu erzwingen kann der Befehl \textbackslash nocite) verwendet werden.
\nocite{eine_doktorarbeit}

Wir zititeren natürlich auch online Quellen: \cite{Wikipedia:2021}.

